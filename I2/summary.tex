\section{Заключение}
В данной работе рассматривался спектр поглощения молекулы йода. Были получены молекулярные константы, построен потенциал Морзе для возбужденного состояния молекулы $\text{I}_2$.

% TODO: точно у нас так же?
Значения этих констант и названия методов, которыми они были получены приведены в таблице 4.
О соответствии полученных результатов табличным также можно судить из таблицы 4.

Анализ спектров при разных ширинах щели показывает, что с уменьшением размера щели увеличивается разрешающая способность прибора. Электронно-колебательную структуру спектра мы начинаем разрешать при ширине щели около 0.5 нм.

